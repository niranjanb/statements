\documentclass[a4paper,11pt,onecolumn]{article}
\usepackage[hscale=0.8,vscale=0.8]{geometry}
\usepackage[parfill]{parskip}
\usepackage{amsmath}
\usepackage{amsfonts}
\usepackage{graphicx}
\usepackage{subfigure}
\usepackage{wrapfig}
\usepackage{amssymb}

\newcommand{\eat}[1]{}

\begin{document}
\include{weld-defns}
\title{Teaching Statement}
\author{Niranjan Balasubramanian}
\maketitle

% high-level philosophy
I enjoy learning, especially in the academic setting. This is one of the main motivations for my academic ambitions. I have spent the better part of the last decade being part of four different academic institutions. From my experience, I have come to realize that the primary purpose of higher education is to provide a platform for students 
to strive for excellence and foster independence in pursuit of their own goals. I view my teaching responsibilities in this broad context well. I intend to draw insights and inspirations from some of the excellent teachers and mentors I have had the good fortune to meet, interact and learn from.

% what topics i will teach and why
In the near future, I will teach graduate and undergraduate courses in information retrieval, natural language processing and artificial intelligence. Strong background in these topics is critical to succeeding in the big data and information economy of today. There are two broad veins of expertise required in these areas: i) Fundamental Computational aspects -- Data structures, algorithms, and handling large scale data. ii) Advanced Theory/Research aspects -- Models, theory, evaluation and experimentation. I am interested in designing courses that straddle both aspects but with a heavier emphasis on computational aspects for undergraduate level and more advanced research aspects for the graduate level.

In the long term, I will develop courses for engaging a broader community of students. My goal here is two-fold: 1) Provide access to computational tools that non-computer science majors can use, and 2) attract women and minorities to computer science. I will teach a project-driven course that gets students excited about solving a real world problem using computers. Articulating the problem, finding the right data and tools provide a walk-through of the main steps involved in solving problems computationally. This provides an attractive non-technical format for introducing some of the core aspects of computer science. 

% Experience
I served as a teaching assistant to Prof. James Allan in his Information Retrieval course. My responsibilities included designing and grading assignments, exams, and evaluating projects. I also delivered two lectures on advanced topics in Information Retrieval. This was a great learning experience for me from practical aspects (setting up access for course materials, writing unambiguous questions etc.) and technical aspects of the material that I had to teach. It taught me the importance of tracking what has been taught before, the need for repetition, and how students respond to enthusiasm.

% Teaching philosophy
I was fortunate to be taught by excellent passionate teachers whom I wish to emulate in my own teaching. One of the courses that I enjoyed most was Theory of Computation taught by Prof. Neil Immerman. The course was structured around crisp lectures that covered the most important aspects in detail and around assignments, where the real learning happened. Apart from his infectious enthusiasm for the subject, the key to the success of his course was what he did not teach in class and how he leveraged the assignments to get students to learn the missing pieces. I have also noticed that successful teachers are always ready to re-think and adapt their course material with feedback from students. 

% Mentoring philosophy
I have worked with mentors who are leading researchers in different fields -- James Allan in Information retrieval, Arun Venkataramani in Systems, and Oren Etzioni in Information extraction. I have received valuable training from these wonderful researchers. The common theme in their mentorship is the strong emphasis on excellence and independence, values I seek to carry forward myself. 

As a mentor I will focus on training students to conduct independent research. I firmly believe in early exposure to doing quality research. While early success inspires confidence, learning to handle failures in research is critical to long term success as a researcher. I will create a supportive environment that aids students to do quality research but also learn how to conduct research. More importantly, I will learn and adapt quickly in my own efforts to be an excellent teacher and mentor.


\end{document}
