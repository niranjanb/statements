\documentclass[a4paper,11pt,onecolumn]{article}
\usepackage[hscale=0.8,vscale=0.8]{geometry}
\usepackage[parfill]{parskip}
\usepackage{amsmath}
\usepackage{amsfonts}
\usepackage{graphicx}
\usepackage{subfigure}
\usepackage{wrapfig}
\usepackage{amssymb}

\newcommand{\eat}[1]{}

\begin{document}
\include{weld-defns}
\title{Teaching Statement}
\author{Niranjan Balasubramanian}
\maketitle

% high-level philosophy
I enjoy learning in all forms but especially in an academic setting. This is one of the main motivations for my academic ambitions. I have spent the better part of the last decade being part of four different academic institutions. Through the course of these years, I have come to realize that the primary purpose of higher education is to provide a platform for students to strive for excellence and foster independence in pursuit of their own goals. I view my teaching responsibilities in this broad context well. I intend to draw insights and inspirations from some of the excellent teachers and mentors I have had the good fortune to meet, interact and learn from.

% what topics i will teach and why
In the near future, I will teach graduate and undergraduate courses in information retrieval, natural language processing and artificial intelligence. Strong background in these topics is critical to succeeding in the big data and information economy of today. There are two broad veins of expertise required in these areas: i) Fundamental Computational aspects -- Data structures, algorithms, and handling large scale data. ii) Advanced Theory/Research aspects -- Models, theory, evaluation and experimentation. I am interested in designing courses that straddle both aspects but with a heavier emphasis on computational aspects for undergraduate level and more advanced research aspects for the graduate level.

In the long term, I will develop courses for engaging a broader community of students. My goal here is two-fold: Provide access to computational tools that non-computer science majors can use and attract women and minorities to computer science. I will teach a project-driven course that gets students excited about solving a real world problem using computers. Articulating the problem, finding the right data, and tools provide a walk-through of the main steps involved in solving problems computationally. This provides an attractive non-technical format for introducing some of the core aspects of computer science. 

I served as a teaching assistant to Prof. James Allan in his Information Retrieval course. My responsibilities included designing and grading assignments, exams, and evaluating projects. I also delivered two lectures on advanced topics in Information Retrieval. This was a great learning experience for me from practical aspects (setting up access for course materials, writing unambiguous questions etc.) and technical aspects of the material that I had to teach. It taught me the importance of tracking what has been taught before, the need for repetition, and how students respond to enthusiasm.

I have had the fortune to be taught by excellent teachers and I seek to incorporate what I have learnt from them in my own teaching. One of the courses that I enjoyed most was Theory of Computation taught by Prof. Neil Immerman. The course was structured around crisp lectures that covered the most important aspects in detail and around assignments, where the real learning happened. Apart from his infectious enthusiasm for the subject, the key to the success of his course was what he did not teach in class and how he leveraged the assignments to get students to learn the missing pieces. I have also noticed that successful teachers are always ready to re-think and adapt their course material with feedback from students. 

%I have had great mentors who have inspired, encouraged, and supported me throughout my graduate school. 
I have the had the good fortune to have worked with mentors who are leading researchers in different fields -- James Allan in Information retrieval, Arun Venkataramani in Systems, and Oren Etzioni in Information extraction. I have learnt many things from these wonderful researchers but the common theme in their mentorship is the strong emphasis they placed on excellence and independence, which I seek to carry forward as a mentor myself. 

I believe that one of my main roles as a research advisor is to enable students to conduct independent research. I also firmly believe in early exposure to doing quality research, while simultaneously acquiring technical expertise via coursework. While early success in research inspires confidence, learning to handle failures in research is critical to long term success as a researcher. I will create a supportive environment that aids students to do quality research but also learn how to conduct research. More importantly, I will seek to learn and adapt quickly in my own efforts to be an excellent teacher and mentor.

%I also firmly believe in early exposure to doing quality research with an emphasis on learning. 

%I carry many lessons from these wonderful researchers with their diverse perspectives and tools that they have used to instill values of excellence and independence. Learning to do independent research requires training on many fronts ranging from technical expertise, writing skills, to identifying which problems are worth solving.  Much like my own mentors, I believe the primary role of a mentor is to inculcate a culture of excellence and independence in students but provide full support when needed. 


%James Allan encouraged students to find projects that they are truly excited by and provided a wonderful supportive environment for conducting independent research.
% fundamentals of information retrieval + information extraction + advanced topics
%I will also design advance courses that focus on the state-of-the-art approaches to algorithms, frameworks, and systems for solving large scale natural language problems such as web search, information extraction, and summarization. My broad research experience in these topics and industrial experience have given me unique perspectives on the abstractions involved in these different applications. 

% outreach efforts (MOOC, Diversity, attracting students to STEM)

%I will seek opportunities to engage in both research and teaching opportunities beyond the department, reaching out to high-school and freshmen in other departments. Cite efforts by Harvey Mudd.

% Mentoring graduate students and undergraduate students.

%In terms of mentoring, I seek to draw inspiration from my own mentors and how they have enriched and supported me throughout. 

% Early focus on research and support. 

% Recognition that each student is unique.

% Push for excellence and independence. 


% Training involves more than specific technical accomplishments (writing, identifying )

% Push for breadth in knowledge and seek synthesis style research efforts. 

% Take teaching seriously. Seek guidance and learn from early experiences.


\end{document}
