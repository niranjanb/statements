\documentclass[a4paper,11pt,onecolumn]{article}
\usepackage[hscale=0.8,vscale=0.8]{geometry}
\usepackage[parfill]{parskip}
\usepackage{amsmath}
\usepackage{amsfonts}
\usepackage{graphicx}
\usepackage{subfigure}
\usepackage{wrapfig}
\usepackage{amssymb}

\newcommand{\eat}[1]{}

\begin{document}
\include{weld-defns}
\title{Teaching Statement}
\author{Niranjan Balasubramanian}
\maketitle

In my view the fundamental purpose of higher education is to push students to strive for excellence and independence in the pursuit of their own goals. I view my teaching responsibilities in this broad context well beyond lectures in classrooms. I intend to draw insights and inspirations from some of the excellent teachers and mentors I have had the good fortune to meet, interact and learn from.

I will teach about basics and advanced topics in Information retrieval, Natural Language Processing, and Artificial Intelligence. These span some of the cutting edge topics required in the big data and information economy of today. These are fundamentally application driven, yet have fostered the tradition of abstractions and algorithms design so prevalent in computer science. 

I will seek opportunities to engage in both research and teaching opportunities beyond the department, reaching out to high-school and freshmen in other departments. I will teach a project-driven course that gets students excited about solving a real world problem using computers. Articulating the problem, finding the right data, and tools provides a direct analogy to how problems are solved computationally. 

My goal is to get students to see that computational thinking and modeling problems is analogous to several skills they may already possess or are training for.

In terms of mentoring, I seek to draw inspiration from my own mentors. I 


\end{document}
