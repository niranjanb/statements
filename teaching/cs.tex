\documentclass[a4paper,11pt,onecolumn]{article}
\usepackage[hscale=0.8,vscale=0.8]{geometry}
\usepackage[parfill]{parskip}
\usepackage{amsmath}
\usepackage{amsfonts}
\usepackage{graphicx}
\usepackage{subfigure}
\usepackage{wrapfig}
\usepackage{amssymb}

\newcommand{\eat}[1]{}

\begin{document}
\include{weld-defns}
\title{Teaching Statement}
\author{Niranjan Balasubramanian}
\maketitle

% high-level philosophy
In my view the fundamental purpose of higher education is to provide students a platform to strive for excellence and foster independence in pursuit of their own goals. I view my teaching responsibilities in this broad context well. I intend to draw insights and inspirations from some of the excellent teachers and mentors I have had the good fortune to meet, interact and learn from.

% what topics i will teach and why
In the near future, I will teach graduate and undergraduate courses in information retrieval, natural language processing and artificial intelligence. Strong background in these topics is critical to succeeding in the big data and information economy of today. These topics present many different approaches f

These are fundamentally application driven, yet have fostered the tradition of abstractions and algorithms design so prevalent in computer science. 

% specific topics
% problem solving with computers
I will teach a project-driven course that gets students excited about solving a real world problem using computers. Articulating the problem, finding the right data, and tools provide a walk-through of the main steps involved in solving problems computationally.

% fundamentals of information retrieval + information extraction + advanced topics
I will also design advance courses that focus on the state-of-the-art approaches to algorithms, frameworks, and systems for solving large scale natural language problems such as web search, information extraction, and summarization. My broad research experience in these topics and industrial experience have given me unique perspectives on the abstractions involved in these different applications. 

% outreach efforts (MOOC, Diversity, attracting students to STEM)
I will seek opportunities to engage in both research and teaching opportunities beyond the department, reaching out to high-school and freshmen in other departments. Cite efforts by Harvey Mudd.

% Mentoring graduate students and undergraduate students.
In terms of mentoring, I seek to draw inspiration from my own mentors. 

% Early focus on research and support. 

% Recognition that each student is unique.

% Push for excellence and independence. 


% Training involves more than specific technical accomplishments (writing, identifying )

% Push for breadth in knowledge and seek synthesis style research efforts. 

% Take teaching seriously. Seek guidance and learn from early experiences.


\end{document}
